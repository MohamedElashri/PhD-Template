\chapter{The detector}
\label{chap:ATLAS}

\chapterquote{I don't like walking around with people thinking I'm doing uncool shit, because there's nothing I'm doing that's uncool.}
{Mohamed Elashri, 1996--Forever}


\section{The \nova experiment}
\label{sec:ATLASDescription}

The \nova detector is really rather something and was largely inspired by dolphins like the one depicted in \FigureRef{fig:dolphin}, in a easily understood way.
\vspace{1cm}

\begin{center}
{\hspace{1mm}\Large\vdots\hspace{1cm}}
\end{center}

\newpage

\begin{figure}
  \begin{center}
      \includegraphics[width=0.5\textheight]{dolphin}
      \caption[The dolphin that inspired \nova]%
              {A dolphin.}
    \label{fig:dolphin}
  \end{center}
\end{figure}

Many dolphins make light work and, yes, the same is true in terms of the
\nova machine at Fermilab in Batavia. Whereas one would like to base all
future detectors on these fun little things, the likelihood is increasingly
low. Recent critics, including Dr Myself, have said, ``I'm not sure I follow what you are talking about'' \cite{buckley2016results}. For the \about{50\percent} of people interested in future designs there are a lot of magazines about.

\newpage

\section{Trigger system}
\label{sec:triggers}
An overview of the \nova triggers are shown in \Table~\ref{tab:TriggerDetails}.

\begin{table}[h]
  \begin{tabular}{lll}
        & FREAKwency      & was it good? \\
    \midrule\\
    Far  & \unit{40}{\MHz} & yeah \\
    Near & \unit{1}{\MHz}  & oh yeah \\

  \end{tabular}
  \caption{The table of triggers.}
  \label{tab:TriggerDetails}
\end{table}
